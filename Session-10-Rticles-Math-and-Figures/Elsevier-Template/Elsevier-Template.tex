\documentclass[]{elsarticle} %review=doublespace preprint=single 5p=2 column
%%% Begin My package additions %%%%%%%%%%%%%%%%%%%
\usepackage[hyphens]{url}

  \journal{An awesome journal} % Sets Journal name


\usepackage{lineno} % add
\providecommand{\tightlist}{%
  \setlength{\itemsep}{0pt}\setlength{\parskip}{0pt}}

\usepackage{graphicx}
\usepackage{booktabs} % book-quality tables
%%%%%%%%%%%%%%%% end my additions to header

\usepackage[T1]{fontenc}
\usepackage{lmodern}
\usepackage{amssymb,amsmath}
\usepackage{ifxetex,ifluatex}
\usepackage{fixltx2e} % provides \textsubscript
% use upquote if available, for straight quotes in verbatim environments
\IfFileExists{upquote.sty}{\usepackage{upquote}}{}
\ifnum 0\ifxetex 1\fi\ifluatex 1\fi=0 % if pdftex
  \usepackage[utf8]{inputenc}
\else % if luatex or xelatex
  \usepackage{fontspec}
  \ifxetex
    \usepackage{xltxtra,xunicode}
  \fi
  \defaultfontfeatures{Mapping=tex-text,Scale=MatchLowercase}
  \newcommand{\euro}{€}
\fi
% use microtype if available
\IfFileExists{microtype.sty}{\usepackage{microtype}}{}
\bibliographystyle{elsarticle-harv}
\ifxetex
  \usepackage[setpagesize=false, % page size defined by xetex
              unicode=false, % unicode breaks when used with xetex
              xetex]{hyperref}
\else
  \usepackage[unicode=true]{hyperref}
\fi
\hypersetup{breaklinks=true,
            bookmarks=true,
            pdfauthor={},
            pdftitle={Short Paper},
            colorlinks=false,
            urlcolor=blue,
            linkcolor=magenta,
            pdfborder={0 0 0}}
\urlstyle{same}  % don't use monospace font for urls

\setcounter{secnumdepth}{0}
% Pandoc toggle for numbering sections (defaults to be off)
\setcounter{secnumdepth}{0}
% Pandoc header



\begin{document}
\begin{frontmatter}

  \title{Short Paper}
    \author[Some Institute of Technology]{Alice Anonymous\corref{c1}}
   \ead{alice@example.com} 
   \cortext[c1]{Corresponding Author}
    \author[Another University]{Bob Security}
   \ead{bob@example.com} 
  
      \address[Some Institute of Technology]{Department, Street, City, State, Zip}
    \address[Another University]{Department, Street, City, State, Zip}
  
  \begin{abstract}
  This is the abstract.
  
  It consists of two paragraphs.
  \end{abstract}
  
 \end{frontmatter}

\hypertarget{methods}{%
\section{Methods}\label{methods}}

\hypertarget{spatial-autocorrelation-and-map-pattern}{%
\subsection{Spatial Autocorrelation and Map
Pattern}\label{spatial-autocorrelation-and-map-pattern}}

Spatial autocorrelation is a condition whereby the value of a variable
at one location is correlated with the value(s) of the same variable at
one or more proximal locations. A tool widely used to measure spatial
autocorrelation is Moran's coefficient of autocorrelation, or \(MC\) for
short. In matrix form, \(MC\) can be formulated as follows:

\begin{equation} 
\label{eq:1}
MC=\frac{n}{\sum_{i}{\sum_{j}{w_{ij}}}}\frac{x'Wx}{x'x}
\end{equation}

where \(x\) is a vector \((n\times1)\) of mean-centered values of a
georeferenced variable, and \(W\) is a spatial weights matrix of
dimensions \((n\times n)\) with elements \(w_{ij}\). The elements of the
spatial weights matrix take non-zero values if locations \(i\) and \(j\)
are deemed to be spatially proximate in some sense, and 0 otherwise. It
can be appreciated that the coefficient is composed to two elements: the
variance of the random variable (i.e., \((x' x)⁄n\)) and its spatial
autocovariance \(\frac{(x'Wx)}{\sum_{i}{\sum_{j}{w_{ij}}}}\). As an
alternative, the numerator of the right-hand term of Equation \ref{eq:1}
can be expressed as follows:

\begin{equation} 
\label{eq:2}
x'\Big(I - \frac{11'}{n}\Big)W\Big(I - \frac{11'}{n}\Big)x
\end{equation}

with \(I\) as the identity matrix of size \(n\times n\) and \(1\) a
conformable vector of ones.

One possible interpretation of spatial autocorrelation is as map
pattern. More concretely, the eigenvalues of the following matrix
represent the range of possible values of \(MC\) given a spatial weights
matrix \(W\), and the extreme eigenvalues are in fact associated with
the minimum and maximum values of \(MC\) for the system of relationships
represented by \(W\):

\begin{equation} 
\label{eq:3}
\Big(I - \frac{11'}{n}\Big)W\Big(I - \frac{11'}{n}\Big)
\end{equation}

A remarkable discovery is that the eigenvectors associated with the
eigenvalues of the matrix in Expression \ref{eq:3} represent a catalogue
of latent map patterns, each with a level of autocorrelation (as
measured by \(MC\)) given by its corresponding eigenvalue. Furthermore,
the patterns represented by the eigenvectors are orthogonal by design,
and so they furnish \(n\) maps that are independent from each other.
Since these map patterns depend only on the spatial weights matrix --
and not the spatial random variable -- they constitute an extensive set
of latent map patterns that can be used in regression analysis as
filters. This is explained next.

\hypertarget{references}{%
\section*{References}\label{references}}
\addcontentsline{toc}{section}{References}


\end{document}


